\section{Introduction}

% Content - Valid Project enquiry proposed - within the context of the specialism chosen if appropriate

My project, named HTool, will attempt to solve the problem posed by the lack of available tools for static analysis, parsing and reverse engineering binaries and firmware files on Apple's iOS, macOS and derived Operating Systems.

This heavily relates to my chosen field of study, Software Engineering, as the resulting artefact will be a complex application that implements numerous algorithms for handling a range of firmware file that use, in some cases, undocumented file formats and structures.

There is currently a distinct lack of available tools and applications for this purpose, particularly for beginners who want to explore the internals of these operating systems. Two of the most popular tools, JTool (\cite{levin-jtool}) and IDA64 (\cite{hexrays-ida64-pro}), are either no longer regularly updated or extremely expensive. With the exception of some highly specific tools for particular functions available on Github, there are no open-source or recent alternatives to JTool. 

In this essay I aim to cover the four primary areas that my project relates to: Mach-O files, iOS Kernel Cache files, iBoot and macOS firmware files. I'll discuss the available literature, the knowledge I have gained from them, and how they have influence my initial project idea. Due to the iOS \& macOS community being fairly small, there is also a lack of research papers - most of the knowledge is shared on blogs, forums and Twitter. In these cases I have outlined the author and why I believe them to be a reputable source. 