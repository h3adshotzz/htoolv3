\subsection{Summary}

% This is the last paragraph of your literature review. In this paragraph, it is important to briefly summarize the main findings from the articles that you reviewed and to point out how your idea and problem were answered or not answered, what you learnt from the process and how your initial ideas for a solution has developed

Throughout this essay I have reviewed a range of sources from a variety of locations, from Tweets and forum posts to technical articles and books. I have covered what I identified at the start as the four main areas the project should focus on - Mach-O files, Kernel Caches, iBoot and macOS. 

I have acquired the following knowledge as a result of this essay:

\begin{enumerate}
	\item There is a distinct lack of resources documenting these firmware file formats, with exception of Mach-O. A potential part of my project that could in part resolve this would be documentation. An in-depth user guide covering how to use the tool, and how the firmware files it can analyse are structured, would fill a knowledge gap that appears to exist - at least with public documentation.
	\item I now have a better understanding of how the Kernel cache and KEXTs work on ARM and Intel-based macOS devices. This is something that prior to this essay I did not have prior knowledge of, and this will prove useful when developing my project. 
	\item I have an understanding of a number of security mitigations, such as Pointer Authentication and Kernel Patch Protection, and how it may be possible to implement algorithms to detect these mitigations in firmware files.
\end{enumerate}

My initial idea was for my project to be a tool that primarily analysed Mach-O and Kernel files, with some support for iBoot and disassembly. However, with the knowledge I know have from this work I believe that this can be expanded, allowing the tool to essentially act as a Swiss army knife for iOS and macOS firmware file analysis. 

Based on the knowledge acquired from this essay, I believe the following features are feasible to implement in my project:

\begin{enumerate}
	\item Mach-O Parser: Header, Load Commands, Segment Commands, Code signing, entitlements.
	\item Kernel Analysis: Device \& Version detection, Kernel cache format detection, KEXT handling (extract a KEXT to its own file, so it can be treated as a regular Mach-O). Security mitigation detection (KPP, PAC, KTRR)
	\item iBoot, SecureROM, etc: Device \& Version detection, iBoot embedded firmware handling.
	\item Command line disassembly of both Mach-O and non-object files.
\end{enumerate}