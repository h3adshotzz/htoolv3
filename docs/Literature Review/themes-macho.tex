\subsection{Mach-O File Format}

Handling Mach-O files is a primary requirement of this project. All iOS and macOS applications are compiled as Mach-O executables and the XNU kernel is also compiled as a Mach-O. Essentially any executable code from the Kernel up will use Mach-O. This means that handling this file format is the first functionality that would need to be implemented into any solution.

In early 2020 I began developing a C library for working with strings, lists, compression, files, Apple's Image4 format and the Mach-O executable file format. The library provides APIs for working with Mach-O files and allow for a binary to be translated into a C structure that can then be used throughout a client program. While the API does provide a Mach-O parser, much is needed to be built on-top of it for the program to meet the specified requirement - and an understanding of the file format is required too. 

I briefly cover the basics of the Mach-O format in an article I wrote on my blog[1]. I cover the Mach-O header, load commands and segment commands, and how these can be handled in C. However, the article doesn't act as a user guide for libhelper, nor does it cover some other aspects of Mach-O files such as the different types, code signing and entitlements as it was only meant to be a quick-to-read introduction.

An article published by Scott Lester, however, does mention these areas in far greater detail. His blog post "A look at Apple Executable Files" covers the file format in much greater detail[2]. More importantly, Lester discusses how code signing and entitlement information is embedded into segment commands - again something Libhelper currently does not have support for. In the past only iOS binaries required code signing, but in more recent versions of macOS it's become something of a soft-requirement and therefore will be an important feature of any application that claims to be able to analyse Mach-O files. These missing features could either be implemented within libhelper directly, or built on-top as part of HTool.

Lester's article is more useful int eh context of the problem as it covers both the Mach-O format in far greater detail than anyone with exception of Apple's developer documentation, and discusses aspects such as code signing and entitlements. While my own article does introduce libhelper, which will be extensively used in HTool, I do not go into sufficient detail in the article regarding the file format.