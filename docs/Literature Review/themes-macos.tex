\subsection{macOS}

% how the theme is relevant to your problem

In the past macOS has typically remained a less restricted and locked-down operating system compared to iOS. Until recently macOS did not have a Kernel Cache, instead the kernel shipped as a single binary part of the Base Image, with extensions also stored as individual binaries on the system partition. They were Mach-O files, just not merged together in a single file like a kernel cache.

% a comparison of the findings from the sources relevant to your problem

There are two types of Mac's that used ARM-based processors. First are the T2 Mac's, and second are the "Apple Silicon" Mac's.

A "T2-enabled Mac" is a macOS device that uses an Intel x86\_64 processor as its primary Application Processor, and an ARM-based chip from an older generation iPhone that is used as a security co-processor. On these machines the T2 handles the Secure Boot chain, Apple Pay and Touch ID. (\cite{duo-labs-t2-intro})

The T2 essentially acts as its own system and handles requests for the main processor. It runs a slimmed-down version of watchOS (which itself is a derivative of iOS), known as BridgeOS. This OS uses a Kernel Cache like on iOS, and therefore a tool designed for analysing firmware files of iOS devices would work just fine with BridgeOS.

The second type is an "Apple Silicon" Mac. These machines use ARM-based processors designed by Apple, like in the iPhone or iPad. There is no need for a T2 chip on these machines, so BridgeOS is not present. As macOS becomes more like iOS, it now also uses Kernel Cache in much the same way. 

Linux kernel developer Hector Martin, who is currently heading a project to port the Linux Kernel to the ARM-based Mac's, notes the following about the Kernel cache on macOS:

"The built-in bootloader actually boots iBoot2 from /System/Volumes/Preboot/.../iBoot.img4, and that then loads the Darwin kernel from /System/Volumes/Preboot/.../com.apple.kernelcaches /kernelcache." (\cite{martin-hackernews-macos-kernel})

From this we can gather than ARM-based Mac's use the same kernel cache as iOS and BridgeOS, and most likely the same format too. Therefore, like BridgeOS, support for the Mac's cache file should be automatic. 

On Intel-based Mac's, however, there is again a different format. Intel-based Mac's do not use a Kernel Cache file, rather they store the plain XNU kernel Mach-O file at one directory, /System/Library/Kernels/kernel, and make use of a concept called "Kernel Collections"

A blog run by an individual who goes by the name "hoakley" discussed the differences between x86 and arm-based mac kernel's in an article entitled "Extensions are moving away from the kernel". (\cite{hoakley-extensions-kernel})

In this article he identifies a number of things. Firstly that there are three types of "Kernel Collections", the "Boot Kext Collection (BKC)", "System Kext Collection (SKC)", and "Auxiliary Kext Collection (AKC)". (\cite{hoakley-extensions-kernel})

\begin{enumerate}
	\item The Boot Kext Collection is used on both ARM and x86\_64 (Intel) Mac's. On ARM, the KEXTs in the BKC are found in the Kernel Cache, and on x86\_64 they are found in the BootKextCollection.kc file under /System/Library/KernelCollections/.
	\item The System Kext Collection is only used on x86\_64, it is not used on ARM-based models. This is found in the same locations as the BKC, and is named SystemKextCollection.kc.
	\item Finally, the Auxiliary Kext Collection is only typically found on x86\_64 systems, where it is managed by the kernelmanagerd service. On ARM platforms, it can only be found if the system is placed in low-security mode, otherwise it's ignored completely.
\end{enumerate}


% what you learnt collectively from these sources

I have gained an understanding of the way kernel extensions work on macOS - something I was unfamiliar of prior to this research. This will be useful when it comes to implementing support for Kernel Collections into HTool. iBoot is the same on all platforms so there isn't much else to learn there.

% how the findings have influenced your initial solutions

The findings have given me both the knowledge and confidence to implement support for analysing and reverse engineering the kernel, kernel cache and kernel collections on macOS for both Intel and ARM architectures into HTool. 

