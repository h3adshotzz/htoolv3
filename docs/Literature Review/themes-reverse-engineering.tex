\subsection{Reverse Engineering}

% how the theme is relevant to your problem

Reverse Engineering is an important technique for discovering security vulnerabilities in systems. While other techniques can be used, such as simply auditing the open-source components of the XNU kernel, its unlikely any vulnerabilities would be discovered without any reverse engineering. 

%\cite{https://www.techtarget.com/searchsoftwarequality/definition/reverse-engineering#:~:text=Reverse%2Dengineering%20is%20often%20used,in%20microprocessors%20using%20reverse%2Dengineering.}

There are already tools available for large reverse engineering efforts, such as attempting to work out how a new kernel extension works - this is where expensive programs like IDA64 are useful, they can analyse, disassemble and decompile entire binaries. However, it is not efficient to load a binary into IDA, especially large ones, just to check a few instructions at a particular address. 

For this use case, having quick command line tools that can disassemble code at a given address is useful. This feature is available in JTool and provided by the developer tool objdump - therefore adding this into HTool would be ideal.

A real-world example of the usefulness of these tools would be the following, where I needed to compare some code from the open-source XNU kernel and a recently-shipped kernel on iOS. I used objdump by specifying the file name, a start address and stop address. The output was the following:

\begin{verbatim}
$ objdump -D kcache-iphone14-5-ios15.arm64 
  --start-address=0xfffffff008310128 
  --stop-address=0xfffffff008310158
	
Disassembly of section __TEXT_EXEC,__text:

fffffff007ba4000 <__text>:
fffffff008310128:  	adrp	x0, 0xfffffff009e14000
fffffff00831012c:  	add	x0, x0, #0
fffffff008310130:  	add	x0, x0, x22
fffffff008310134:  	sub	x0, x0, x23
fffffff008310138:  	bl	0xfffffff00831d67c
fffffff00831013c:  	mov	sp, x0
fffffff008310140:  	adrp	x0, 0xfffffff009e0c000
fffffff008310144:  	add	x0, x0, #0
fffffff008310148:  	add	x0, x0, x22
fffffff00831014c:  	sub	x0, x0, x23
fffffff008310150:  	msr	SPSel, #0
fffffff008310154:  	mov	sp, x0
\end{verbatim}

There are two possible options for implementing this kind of functionality. Option 1 would be to develop and implement a completely new and custom disassembly framework, something that is a significant challenge and would require extensive knowledge of the ARM architecture and the ARM reference manual \cite{}.

The second option would be to make use of an existing framework such as Capstone Engine.

% a comparison of the findings from the sources relevant to your problem


% what you learnt collectively from these sources


% how the findings have influenced your initial solutions





https://www.techtarget.com/searchsoftwarequality/definition/reverse-engineering#:~:text=Reverse%2Dengineering%20is%20often%20used,in%20microprocessors%20using%20reverse%2Dengineering.




